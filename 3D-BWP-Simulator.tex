\documentclass[11pt,a4paper]{report}
\usepackage[bahasa]{babel} %bahasa indonesia
\usepackage{graphicx} %gambar
\usepackage{sectsty}
\usepackage{setspace}
\usepackage{indentfirst}
\usepackage{url}

\renewcommand{\baselinestretch}{1.5} 
\chapterfont{\centering}

\author{Albert Kamord (2011730077)}
\title{3D-BWP-Simulator}
\begin{document}
\maketitle
\begin{abstract}

\indent Banyak pemain-pemain game zaman sekarang sudah tidak tertarik lagi dengan permainan tradisional. Dewasa ini, banyak orang yang lebih memilih permainan digital. Banyak permainan-permainan tradisional yang sudah dibuat versi digitalnya contohnya Sudoku. Walaupun sudah dibuat versi digitalnya, peminat permainan tradisional semakin menurun saja.

\indent Di dalam tulisan ini, akan disampaikan solusi untuk menarik kembali perhatian para \textit{gamers} dengan cara membuat game Sudoku di platform 3D. Game 3D ini akan dikaitkan dengan Block World Problem yang telah diketahui banyak orang. \\
\\
Kata kunci : \textbf{3D block world problem}
\end{abstract}

\tableofcontents \newpage 	% Daftar isi
\listoffigures \newpage 	% Daftar gambar


\chapter{Pendahuluan} %Pendahuluan
\section{Latar Belakang Masalah}
\indent Di zaman modern ini, banyak game digital yang sudah dibuat. Permainan-permainan sebelum era modern sudah dilupakan, contohnya Sudoku. Sudoku yang ada di zaman sekarang juga sudah dibuat versi digitalnya, tetapi peminat permainan ini semakin menurun saja. Sudoku pernah populer di seluruh dunia, terutama di kalangan orang-orang yang ingin mencoba kemampuan pikirannya.

\indent Perubahan zaman tentu saja menyebabkan banyak perubahan di dunia permainan juga. Dari permainan tradisional ke permainan yang menggunakan \textit{console}. Tentu saja permainan console juga mengalami perubahan. Awalnya, hanya permainan simple seperti tetris, kemudian permainan yang lebih kompleks seperti Super Mario Bros, sampai sekarang yang telah menggunakan teknologi-teknologi canggih untuk membuat sebuah game, contohnya Call of Duty.

\indent Banyak game digital yang telah dibuat dapat dibagi menjadi dua berdasarkan sudut pandang pemain, yaitu game 2D(dua dimensi) dan game 3D(tiga dimensi). Game 3D tentu saja lebih menarik perhatian daripada game 2D. Di dalam tulisan ini, akan disampaikan solusi untuk menarik perhatian pemain-pemain game dalam memainkan game Sudoku yang dikaitkan dengan Block World Problem dengan menggunakan Platform 3D.

\section{Rumusan Masalah}
Rumusan masalah yang terdapat di tulisan ini adalah:
\begin{enumerate}
	\item Mengapa jumlah peminat Sudoku semakin menurun?
	\item Apa solusi yang tepat untuk menaikkan jumlah peminat Sudoku?
\end{enumerate}

\section{Tujuan}
Tujuan dari tulisan ini adalah : Menarik perhatian gamers untuk mencoba memainkan Sudoku yang dikaitkan dengan BWP dengan pandangan 3D

\chapter{Isi} %isi
\section{BWP}
Block World Problem


\chapter{Penutup} %Penutup
\section{Kesimpulan}
Kesimpulan
\section{Saran}
Saran

\begin{thebibliography}{9}

%@article{felgenhauer2006mathematics,
  %title={Mathematics of sudoku I},
  %author={Felgenhauer, Bertram and Jarvis, Frazer},
  %journal={Mathematical Spectrum},
  %volume={39},
  %number={1},
  %pages={15--22},
  %year={2006},
  %publisher={[Oxford, Eng.] Oxford University Press.}
%}

\bibitem{}
  Jiawei Han, Micheline Kamber
  \emph{Data Mining Concept and Techniques}.
  Morgan Kaufmann Publishers,
  2nd Edition,
  2006.

\end{thebibliography}
\end{document}
 